\documentclass[a4paper, 12pt]{extarticle}
\usepackage{../GS6}
\sloppy

\begin{document}
		\def \nocredits {}
		\def \LineC {Конспект по дисциплине}
		\def \LineD {Защищённые информационные системы}
		\def \LineE {}
		\def \LineF {}
		\def \LineG {}
		\def \LineH {}

\maketitle

	\section{Термины и определения}

		Функциональные качества  технических устройств, в.т.ч. информационных систем, безопасность данных систем в большой степени зависит от их надёжности.
		Под ИС мы будем понимать сложную программно-аппаратную систему, включающую в свой состав эргатические звенья, технические средства и ПО.

		Говоря о надёжности информационных систем следует учитывать две основные составляющие - надёжность аппаратных средств и надёжность ПО.

		Теория надёжности опирается на перечень различных ГОСТов. Основной ГОСТ 27002-89.
		\begin{itemize}
			\item Под объектом по теории надёжности подразумевается техническое изделие определённого назначения, рассматриваемое в периоды проектирования, производства , испытания и эксплуатации. Объектами также могу быть системы и их элементы.
			\item Под системой подразумевается объект, представляющий собой совокупность элементов, связанных между собой определёнными отношениями, и взаимодействующих таким образом, чтобы обеспечить выполнение системой некоторых достаточно сложных функций.
		\end{itemize}

		С точки зрения надёжности выделяют 4 состояния объекта
		\begin{enumerate}
			\item Исправность - состояние объекта, при котором он соответствует всем требованиям, установленным в нормативно-технической документации (обычное состояние защищённой ИС).
			\item Неисправность - состояние объекта, при котором он не соответствует хотябы одному из требований, установленных нормативно-технической документации (ЗИС - угроза безопасности)
			\item Работоспособность - состояние объекта, при котором он способен выполнять заданные функции, сохраняя значения основных параметров, установленных в нормативно-технической документации
			\item Неработоспособность - состояние объекта, при котором значение хотя-бы одного из параметров, характеризующего способность исполнять заданные функции не соответствует требованиям, установленным в нормативно-технической документации (DoS для ЗИС)
		\end{enumerate}


		С точки зрения теории надёжности различают 6 различных переходов объекта в заданные состояния:
		\begin{enumerate}
			\item Повреждение - событие, заключающееся в нарушении исправности объекта при сохранении его работоспособности
			\item Отказ - событие, заключающееся в нарушении работоспособности объекта
			\item Критерий отказа - отличительный признак или совокупность признаков, согласно которым устанавливается факт отказа
			\item Восстановление - процесс обнаружения и устранения отказа с целью восстановления объектом его работоспособности
		\end{enumerate}

		Восстанавливаемый обьект - объект, работоспособность которого после отказа подлежит восстановлению в заданных условиях

		Невосстанавливаемый - объект, работоспособность которого после отказа не подлежит восстановлению в заданных условиях

		Рассмотрим следующие временные характеристики:
		\begin{itemize}
			\item Наработка - Продолжительность работы объекта. Объект может работать как непрерывно, так и в временными интервалами. Во втором случае будет учитываться суммарная наработка.
			\item Технический ресурс - наработка объекта от начала его эксплуатации до достижения предельного состояния.
			\item Срок службы объекта - календарная продолжительность эксплуатации объекта от её начала или возобновления после ремонта до наступления предельного состояния.
			\item Эксплуатация объекта - стадия его существования в распоряжении потребителя при условии применения объекта по назначению, что может чередоваться с хранением, транспортировкой, техобслуживанием и ремонтом, если это осуществляется потребителем.
		\end{itemize}

		\term{Надёжность} - (по ГОСТ 27002) - свойство объекта сохранять во времени в установленных пределах значения всех параметров, характеризующих способность выполнять требуемые функции в заданных режимах и условиях применения.

		С точки зрения ИБ надёжность представляет собой способность ИС противостоять внешним или внутренним угрозам ИБ.

	\section{Факторы, определяющие надёжность ИС}
		Для построения ИС используются различные типы обеспечения: экономическое, временное, организационное, структурное, технологическое, эксплуатационное, социальное, эргатическое, алгоритмическое, синтаксическое и семантическое.

		Под обеспечением можно характеризовать совокупность факторов, способствующих достижению заданной цели.

		Организационное, временное и экономическое обеспечение, обуславливаемое необходимостью материальных и временных затрат используется для поддержания достоверности результатов работы ИС

		Структурное обеспечение ИБ должно обеспечивать надёжность функционирования комплексов и эргатических звеньев, а также ИС в целом.
		Здесь обосновывается рациональное построение ИС, её структуры, зависящее от выбора структуры техпроцесса преобразования информации, обеспечения взаимосвязи между отдельными элементами системы, резервированию и использованию устройств, осуществляющих процедуры контроля.

		Надёжность и технологическое обеспечения связана с выбором для конструктивных решений отдельных комплексов, входящих в состав системы, технологий и протоколов реализации информационных процессов.

		Эргатичесекое обеспечение включает комплекс фактов, связанных с рациональной организацией работы человека в системе - правильное расположение функций между людьми и технологическими устройствами.

		Надёжность алгоритмического обеспечения связана с обеспечением высокого качества и безошибочности алгоритмов и программ преобразования информации и реализации контроля достоверности информации.

		Информационное. синтаксическое и семантическое обеспечение должно обеспечить специальную информационную избыточность, избыточность данных и смысловую избыточность, обуславливающей возможность поведения контроля достоверности информации.



%28.09.16
		(hint: http://sdo2.irgups.ru/course/view.php?id=69)

		Виды ошибок:
		\begin{itemize}
			\item 1
			\item Ошибки совместимости (с ОС)
			\item Ошибки сопряжения
		\end{itemize}

		\subsection{Основные показатели надёжности ПО}
		Если рассматривать отказавшее программное обеспечение без учёта его восстановления, а также случайный характер отказов - то модель надёжности будет принимать вид невосстанавливаемой информационной системы и, следовательно, основными показателями будут следующие величины:
		\begin{itemize}
			\item P(t) - вероятность, что ошибки программы не проявятся в интервале (0;t)
			\item Вероятность события отказа ПО - q(t) - вероятность, что ошибки программы проявятся в интервале (0;t)
			\item Интенсивность отказа $\lambda(t)$
			\item Время наработки на отказ $T$
		\end{itemize}

		При определении характеристик надёжности ПО учитывается тот факт, что возникающие при работе программ ошибки устраняются, количество ошибок уменьшается $\Rightarrow$ интенсивность отказов уменьшается, и наработка на отказ должна увеличиваться.

	 В связи с такими предположениями рассматривается несколько моделей надежности про-
граммного обеспечения:

	\subsubsection{Модель   с  дискретно понижающейся частотой ошибок ПО}
		(ref: http://sdo2.irgups.ru/pluginfile.php/41173/mod\_resource/content/1/Лекция\%20№\%2011.pdf)

		В этой модели полагается, что интенсивность отказов $\lambda(t)$ является постоянной величиной до обнаружения возникшей ошибки. После этого значение интенсивности уменьшается, и данная величина становится опять постоянной.

		В данной модели интенсивность отказов $\lambda(t)$ можно выразить формулой $\lambda(t) = k(M-i) = \lambda$, где $M$ - первоначальное число ошибок; $i$ -  число обнаруженных ошибок, зависящее от времени $t$; $k$ - некоторая константа.

		\imgnh{IMG/3/02},0.8

		Плотность распределения времени обнаружения $i$ - й ошибки в момент времени $t_i$
 определяется соотношением $f(t_i) =\lambda_i e^{-\lambda_it_i}$, а параметры $k$ и $M$ будут устанавливаться на основе наблюдения интервалов между ошибками.

 		На практике же условия данной модели не соблюдаются, так как при устранении ошибок интенсивность отказов уменьшается на одну и ту же величину $k$, но разные ошибки имеют разный вес. Довольно часто возникают ситуации, когда исправление старых ошибок вызывает новые ошибки. Не всегда удаётся устранить причину ошибки, и ПО продолжают использовать, так как при других исходных данных ошибка может себя и не проявлять.

	\subsubsection{	Модель с дискретным увеличением времени наработки на отказ}
	\imgnh{IMG/3/03},0.8
	\imgnh{IMG/3/04},0.8
	Рисунок показывает временные интервалы наработки на отказ.
	Величины $t_1$,$t_2$,$t_3$... $t_m$ - случайные моменты возникновения  первого, второго, третьего и так далее – $m$-го отказов
	Величины $t^{(1)}$,$t^{(2)}$,$t^{(3)}$... $t^{(m)}$ - случайные интервалы времени между возникновением соседних отказов.

	Пусть первая ошибка, появившаяся при работе программы, происходит в момент времени $t_1$ и была устранена. Наработка до первого отказа и возникшей ошибки равна интервалу времени $t^{(1)}$, так как после перезапуска системы у нас возникает вторая ошибка через интервал времени $t_2$, c наработкой системы на отказ, равной $t^{(2)}$. В соответствии с предположением, этот интервал больше, чем $\Delta t_1$, так как после перезапуска программа проработала время до возникновения первой устраненной ошибки, а затем продолжила работу до
новой второй ошибки.

	Тогда $t^{(2)}=t^{(1)}+\Delta t^{(2)}$, где $\Delta t^{(2)}$ - дополнение до $\delta t^{(1)}$.

	Случайное время возникновения ошибки $i-1$ в интервал времени $t_i$ всегда отсчитывается с момента времени $t=0$. Время на ликвидацию ошибки в расчёт не берётся. В этом случае для всех случайных моментов времени возникновения ошибки и временных интервалов между соседними ошибками можно записать:
\begin{align*}
	t_1 &= t^{(1)}\\
	t_2 &= t^{(1)} + t^{(1)} + \Delta t^{(2)}\\
	t_2 &= t^{(1)} + t^{(1)} + \Delta t^{(2)} + t^{(1)} + \Delta t^{(2)} +\Delta t^{(3)}\\
	\ldots &	\\
	t_m &= m \cdot t^{(1)} + (m-1)\cdot\Delta t^{(2)} + (m-2)\Delta t^{(3)} +... +2\Delta t^{(m-1)} +\Delta t^{(m)}
\end{align*}

	Учитывая, что от момента $t_0$ до момента $t_1$ не выявлено ни одной ошибки и что интервал $t_1$ сравнительно невелик, так как ошибки программы в начале эксплуатации происходят довольно часто можно представить интервал наработки на отказ как $\delta t_i$

	Как видим, с последующим запуском программы после обнаружения и устранения ошибки временной интервал между соседними отказами постоянно увеличивается. Следовательно, увеличивается средняя наработка на отказ. Величину наработки на отказ программы можно оценить как:
	$$t_{\mbox{ср}} = \frac{\sum\limits_{i-1}^m t^{(i)}}{m}$$

	Теперь рассмотрим значения $\Delta t_i$
	$t^{(2)}$

	Естественно, для любого i большего ~(мудак, бля)~ можем записать

	$t^{(m)} = \sum\limits_{i-1}^m \Delta t^{(i)}$

	Можем заметить. что $\Delta t^{(i)}$ равна матожиданию $t_m$, а
	$t^{(m)}=M[t^{(m)}]$

	Но для любого i это матожидание равно

	\imgnh{IMG/3/06},0.8

	Приводя это упрощение, можем выразить среднее время наработки на отказ
	\imgnh{IMG/3/07},0.8

	То-же самое мы можем провести с атаками на информационные системы.

	Отсюда видно, что с увеличением числа ошибок увеличивается и средняя наработка между двумя отказами. Рассмотрим среднюю наработку до возникновения $m$-го отказа
	$t^{(m)}_{\mbox{ср}} = $
	\imgnh{IMG/3/08},0.8

	Как и в предыдущем случае здесь видно, что средняя наработка до отказа возрастает с увеличением числа отказов.  Оценки матожидания и дисперсии для данных величин выглядят следующим образом:
	\imgnh{IMG/3/09},0.8

	$\- \sigma^2_{\Delta t} = \sdots$
	 Где $M_n$ - это число отказов за интервал времени от 0 до M.


	new

	Интенсивность же отказоа считается непрерывной функцией, пропорциональной числу оставшихся ошибок
	$m_0(\tau) = M - m(\Еau)$
	$\lamивф(\Tau) = Cm)o(\Tau)$ где C - коэфф пропорциональности быстродействия системы. \sim

	Пусть в процессе исправления ошибок новые ошибки не появляются. Слежовательно  интенисвность исправления ошибок будет равна интенисвночти их обнаружения
	$\Frac{dm(\Tau)}{d\Tau}=\lambda(\Tau)$

	Если сопоставим два пред выражения, с небольшой поправкой получим


	С учётом данного выражения, число исправленных ошибок примет данный вид" $m(\Tau) = M[1-e^{-c\Tau}]$

	Если мы будем выражать время наработки на отказ $T_0 = \frac{1}{\lanmbda(\Tau)}$

	$T_0 = \frac{1}{CM}e^{C\Tau}$
	Также стоит учтывать величину времени среднего периодан наработки на отказ перед тестированием $T_0 = \frac{1}{CM};T_0 = T_{0M} e^{\frac{\Tau}{CM}{MT_{0m}}}$

	Отсюда следует, что среднее время наработки на отказ увеличивается, по мере выявлния и исправления ошибок




	\subsubsection{Характеристики надёжности при хранении информациии}

	В процессе работы информационной системы ошибки, возникающие при хранении информации, делятся на неустранимые и корректируемые. ПРичинами нестранимых оштибок являютмся дефекты физического характера. Они заключаются в том, что некоторые элементы микросхем перестают изменять своё состояние при записи, в соедствие чего считываемый с них код не соответвствует переданному при записи.

	Неустранимые ошиби явяются следствием дефектов производмственного характера, старения или эксплуатации. Корректируемые же ошибки номсяот случайный характери и не явялются результатом неивправенитсти модуля. Они вызывются причинами, начиная от помех в цепях электропитания, внегшней радиацией и заканчивая тмперекатурной нестабильностью в работе микросхем.

	В осоврем енных ВС применяются алгоритмы поиска и восстановления ошиок, но есть и невоссттанавливаемые системы Например, модули памяти относятся невосстанавиваемым системой, и модель. ПРи этом следует иметь ввиду, что современные модули ОЗУ имеют очень высокую надёжность с MTBF около 100000 часов.

	Регистровая кэш-память представляет собой буфер между ОЗУ и ЦП, ио бладает высокой скоростьб передачи данных и сравнительно неботлшим объёмом. В кэш-памяти кратковременно хранятся копии блоков данных тех областей ОЗУ, к которым выполнялись последние обоащения и вероятны обращения в ближайшие такы работы. Также невосстанавливаемый.

	Жёсткие диски - практика эксплуатации показывает, что среднее время наработки на отказ растёт, и в настоящий момент достигает около 1500000 часов. Введём некоторые допущения: пусть НЖМД находятся в периоде нормальеной эксплуатации, выпущены одной серией и запущены одновременно. Режим работы 24x7x365ю Кроме того, предположим, что НЭЖМД поставлены на эксплуатауцию после прохождения периода приработки. Это значит, что дефекты, связанные с проектированием, монтажом не учитываются. Допущение о круглосуточной работе даёт основания полагать, что остановки накопителя информации будут связаны с отключением электропитания. С одной стороны, эти остановки приводят к ускоренному иззносу оборудования, а с другой стороны постоянная работа винчестерва будет причиной сравнитеьного быстрого ищноса механики. Будем считать, что любой отказ вHDD cfzpfy c gjkyjq kb,j xfcnbxyjq gjnthtq/ искажением инфориации. В связи с принятыми довущиниями, будем сичтать MTBF современных HDD = 60000ч.

	Тогда для периода нормальной эксплуатации интенсивность отказов будет равна $\lam$
	Теперь определим величину вероятности безотказной работы винчестеров через год, 2, 3, 4, 5. Пустьна восстановлениеинформации после отказа винчестеров тратится 10\% годового бюджета времени, тогда $$t_{р.д.} = 328.5$$ - время работы в течение 1 года.

	ТОгда вероятноть безотказной работы для первого года $P_i = e^{-\lambdat_2(1)} = 0,875$
	$P_2 = 0,764$;
	$P_3 = 0,669$;
	$P_4 = 0,571$;
	$P_5 = 0,504$;

	Данные результаты показывают, что в среднем, каждый год отказыает по 13% устройств. Можно предполагать, что чем дольше диск работает, то тем более ценная информация на нём хранится. оэтому ущерб, наносиый при выходе из строя винчесера через 5 лет более ощутим, чем при отказе через год.




























\end{document}