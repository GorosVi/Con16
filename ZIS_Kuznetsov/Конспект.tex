\documentclass[a4paper, 12pt]{extarticle}
\usepackage{../GS6}
\sloppy

\begin{document}
		\def \nocredits {}
		\def \LineC {Конспект по дисциплине}
		\def \LineD {Защищённые информационные системы}
		\def \LineE {}
		\def \LineF {}
		\def \LineG {}
		\def \LineH {}
		
\maketitle

	\section{Термины и определения}

		Функциональные качества  технических устройств, в.т.ч. информационных систем, безопасность данных систем в большой степени зависит от их надёжности.
		Под ИС мы будем понимать сложную программно-аппаратную систему, включающую в свой состав эргатические звенья, технические средства и ПО.

		Говоря о надёжности информационных систем следует учитывать две основные составляющие - надёжность аппаратных средств и надёжность ПО.

		Теория надёжности опирается на перечень различных ГОСТов. Основной ГОСТ 27002-89.
		\begin{itemize}
			\item Под объектом по теории надёжности подразумевается техническое изделие определённого назначения, рассматриваемое в периоды проектирования, производства , испытания и эксплуатации. Объектами также могу быть системы и их элементы.
			\item Под системой подразумевается объект, представляющий собой совокупность элементов, связанных между собой определёнными отношениями, и взаимодействующих таким образом, чтобы обеспечить выполнение системой некоторых достаточно сложных функций.
		\end{itemize}

		С точки зрения надёжности выделяют 4 состояния объекта
		\begin{enumerate}
			\item Исправность - состояние объекта, при котором он соответствует всем требованиям, установленным в нормативно-технической документации (обычное состояние защищённой ИС).
			\item Неисправность - состояние объекта, при котором он не соответствует хотябы одному из требований, установленных нормативно-технической документации (ЗИС - угроза безопасности)
			\item Работоспособность - состояние объекта, при котором он способен выполнять заданные функции, сохраняя значения основных параметров, установленных в нормативно-технической документации
			\item Неработоспособность - состояние объекта, при котором значение хотя-бы одного из параметров, характеризующего способность исполнять заданные функции не соответствует требованиям, установленным в нормативно-технической документации (DoS для ЗИС)
		\end{enumerate}


		С точки зрения теории надёжности различают 6 различных переходов объекта в заданные состояния:
		\begin{enumerate}
			\item Повреждение - событие, заключающееся в нарушении исправности объекта при сохранении его работоспособности
			\item Отказ - событие, заключающееся в нарушении работоспособности объекта
			\item Критерий отказа - отличительный признак или совокупность признаков, согласно которым устанавливается факт отказа
			\item Восстановление - процесс обнаружения и устранения отказа с целью восстановления объектом его работоспособности
		\end{enumerate}

		Восстанавливаемый обьект - объект, работоспособность которого после отказа подлежит восстановлению в заданных условиях
		
		Невосстанавливаемый - объект, работоспособность которого после отказа не подлежит восстановлению в заданных условиях

		Рассмотрим следующие временные характеристики:
		\begin{itemize}
			\item Наработка - Продолжительность работы объекта. Объект может работать как непрерывно, так и в временными интервалами. Во втором случае будет учитываться суммарная наработка.
			\item Технический ресурс - наработка объекта от начала его эксплуатации до достижения предельного состояния.
			\item Срок службы объекта - календарная продолжительность эксплуатации объекта от её начала или возобновления после ремонта до наступления предельного состояния.
			\item Эксплуатация объекта - стадия его существования в распоряжении потребителя при условии применения объекта по назначению, что может чередоваться с хранением, транспортировкой, техобслуживанием и ремонтом, если это осуществляется потребителем.
		\end{itemize}

		\term{Надёжность} - (по ГОСТ 27002) - свойство объекта сохранять во времени в установленных пределах значения всех параметров, характеризующих способность выполнять требуемые функции в заданных режимах и условиях применения.

		С точки зрения ИБ надёжность представляет собой способность ИС противостоять внешним или внутренним угрозам ИБ.

	\section{Факторы, определяющие надёжность ИС}
		Для построения ИС используются различные типы обеспечения: экономическое, временное, организационное, структурное, технологическое, эксплуатационное, социальное, эргатическое, алгоритмическое, синтаксическое и семантическое.

		Под обеспечением можно характеризовать совокупность факторов, способствующих достижению заданной цели.

		Организационное, временное и экономическое обеспечение, обуславливаемое необходимостью материальных и временных затрат используется для поддержания достоверности результатов работы ИС

		Структурное обеспечение ИБ должно обеспечивать надёжность функционирования комплексов и эргатических звеньев, а также ИС в целом.
		Здесь обосновывается рациональное построение ИС, её структуры, зависящее от выбора структуры техпроцесса преобразования информации, обеспечения взаимосвязи между отдельными элементами системы, резервированию и использованию устройств, осуществляющих процедуры контроля.

		Надёжность и технологическое обеспечения связана с выбором для конструктивных решений отдельных комплексов, входящих в состав системы, технологий и протоколов реализации информационных процессов.

		Эргатичесекое обеспечение включает комплекс фактов, связанных с рациональной организацией работы человека в системе - правильное расположение функций между людьми и технологическими устройствами.

		Надёжность алгоритмического обеспечения связана с обеспечением высокого качества и безошибочности алгоритмов и программ преобразования информации и реализации контроля достоверности информации.

		Информационное. синтаксическое и семантическое обеспечение должно обеспечить специальную информационную избыточность, избыточность данных и смысловую избыточность, обуславливающей возможность поведения контроля достоверности информации.



\end{document}