\documentclass[a4paper,12pt]{report}
\usepackage{../GS6}
\sloppy

\begin{document}
	\def \nocredits {}
		\def \LineE {Конспект по дисциплине}
		\def \LineF {Теоретические основы КБ}

\maketitle

	\subsection{Зикратов Игорь Алексеевич}

	\subsection{Угрозы ИБ. Методы оценки уязвимостей.}
		% доктрина ИБ (~ 9 сен 1991)
		Закон об информации, защите объектов информатизации - определение информации -> сведения вне зависимости от формы представления.

		\term{Информация} - совокупность данных и соответствующих методов её обработки.

		\term{Информация} - сведения, необходимые для принятия решения (третий вариант определения).

		\term{Сообщение} - совокупность зарегистрированных данных.

		\term{Информация} есть информация, а не материя или энергия.

		\term{Информационная сфера} -  информация, информационная инфраструктура, информационные объекты (носители, потребители), совокупность регламентирующих отношений.
		\imgnh{IMG/1/21},0.8

		Часто информационные объекты разделяют на критически важные, важные и остальные объекты

		Критически важный информационный объект - информационный объект, вывод из строя которого приводит к выходу из строя информационной системы

		Важный информационный объект - информационный объект, вывод из строя которого  приводит к утрате информационной системой некоторых её функций.

		Иной информационный объект - информационный объект, вывод из строя которого не влечёт существенных изменений для информационной системы

		Информационный объект содержит связанные ресурсы, структуры и программы

		\imgnh{IMG/1/20},0.8

		\term{Информационная безопасность} - такое состояние рассматриваемой информационной системы, при котором она, с одной стороны, способна противостоять дестабилизирующему воздействию внешних и внутренних информационных угроз, а с другой стороны, её функционирование не создаёт информационных угроз окружающей среде.

		\subsubsection{Угрозы}

			\term{Угроза} - совокупность факторов и условий, создающих опасность нарушения ИБ организации, вызывающую (или способную вызвать) негативные последствия для организации (ГОСТ 53114-2008)
			\imgnh{IMG/1/01},0.8

			фото - угроза нарушения конфиденциальности

			\term{Канал утечки по времени} (НСД)
			\imgnh{IMG/1/02},0.8
			\imgnh{IMG/1/22},0.8

			\term{Угроза нарушения целостности информации} - незаконные модификация или уничтожение информации.
			\imgnh{IMG/1/03},0.8

		\subsubsection{Классификация угроз по природе происхождения}
			\imgnh{IMG/1/04},0.8

		\subsubsection{Каналы НСД}
			\imgnh{IMG/1/05},0.8
			Классификация взято из учебника (забытого =)

		\subsubsection{Уязвимость ИС}
			\imgnh{IMG/1/06},0.8
			Отличие от угрозы в том, что угроза характеризует внешнюю среду, а уязвимость характерна для конкретной системы.
			\term{Уязвимость} - свойство, обусловливающее возможность реализации угроз безопасности для обрабатываемой информации.

		\subsubsection{Оценка угроз и уязвимостей:} прямая экспертная оценка, статистический анализ, факторный анализ

		\subsubsection{Экспертные оценки}
		Методы поиска решений не поддающихся формализации задач, основанные на суждениях (оценках) экспертов.
		\imgnh{IMG/1/07},0.8

			Способы работы с экспертами:
			\begin{itemize}
				\item Интервьюирование
				\item Анкетирование
			\end{itemize}

			Форма выражения оценки может быть явной  и неявной

			Неявное выражение состоит в том, что эксперт ранжирует оцениваемые элементы (объекты, явления) по степени их важности. Возможен вариант выделения групп с последующим ранжированием объектов внутри группы.

			Явная оценка - эксперт даёт элементам лингвистические или количественные оценки (напр, опасно / безопасно / очень опасно).
			Оценка считается согласованной, если коэффициент конкордации $W$ (коэфф.  согласованности экспертов) больше 0.75

			$$
			W = \frac{12}{d^2(m^3 - m)}S
			$$
			\imgnh{IMG/1/08},0.8

			В случае, когда в ранжировке есть одинаковые значения - применяют другую формулу (некоторые факторы оценены на одинаковый ранг)
			$$
			W = \frac{12 S}{d^2(m^3 - m) - d \sum\limits^d_{s=1}T_s}
			$$
			\imgnh{IMG/1/09},0.8
			\imgnh{IMG/1/10},0.8

			Если с согласованностью проблемы - может применяться метод делфи (aka обратная связь) - эксперты знакомится с анонимными аргументированными обоснованиями суждений коллег.

	\subsection{Оценка привлекательности активов для потенциального злоумышленника}
		\imgnh{IMG/1/11},0.8
		$$
		\gamma = \frac{P^U B_0}{C_0}
		$$
		$C_0$ - стоимость затрат злоумышленника на реализацию угрозы
		$B_0$ - профит злоумышленника при реализации угрозы
		$P^U$ - вероятность реализации угрозы

		\imgnh{IMG/1/12},0.8
		Конечная формула
		\imgnh{IMG/1/13},0.8

	\subsection{Методика оценки уязвимости}
		Малюк, Герасименко

		\imgnh{IMG/1/14},0.8

		Условия НСД -
		\begin{itemize}
			\item Нарушитель должен получить доступ в контролируемую зону
			\item Во время нахождения нарушителя в КЗ должен проявиться соответствующий канал НСД
			\item Нарушитель должен обладать средствами для использования канала НСД
			\item В канале НСД в момент  доступа нарушителя должна быть защищаемая информация.
		\end{itemize}
		Эти события случайны.
		%15
		\imgnh{IMG/1/16},0.8

		ГосТехКомиссия - 4 категории нарушителя - Неподготовленный( пользователей) -> Разработчик (Максимальный доступ и возможности)

		\imgnh{IMG/1/17},0.8
		$L$ - количество зон;
		$P^{\mbox{б}}$ - базовая вероятность реализации угрозы (?)

		Пусть $K^*$ есть интересующее нас подмножество из полного множества потенциально возможных нарушителей. Тогда величина $P_{ij\{K^*\}}$ (третья формула) есть ничто иное, как вероятность нарушения защищённости информации указанным подмножеством нарушителей по $j$-м фактору в $i$-м компоненте системы
		\begin{itemize}
			\item $P_{ij\{K^*\}}$		Наиболее опасный нарушитель
			\item $P_{ik\{J^*\}}$		Наиболее небезопасный канал
			\item $P_{jk\{I^*\}}$		Наиболее уязвимый компонент
		\end{itemize}

		REF: DSEC классификация угроз

		\subsection{Оценка риска ИБ}
		\imgnh{IMG/1/18},0.8

		Риск - вероятность реализации угрозы и нанесение ущерба

	\subsection{Семинар}
		\imgnh{IMG/1/19},0.8

		Лекция - через НЕЧ неделю
		На следующей НЕЧ - вопросы семинара

		Пункт 2 - индивидуальный задания на зачёт.
		1 половина  - ПК
		2 половина - Смартфон
		(по Коновалову)
		Итог  - таким  образом видно, что вероятность ( угроза является более/менее актуальной)

		Время - 15-00






\end{document}