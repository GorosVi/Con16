\documentclass[a4paper,12pt]{report}
\usepackage{../GS6}
\sloppy

\begin{document}
	\def \nocredits {}
		\def \LineE {Конспект по дисциплине}
		\def \LineF {Теоретические основы КБ}

\maketitle

	\subsection{Зикратов Игорь Алексеевич}

	\subsection{Угрозы ИБ. Методы оценки уязвимостей.}
		% доктрина ИБ (~ 9 сен 1991)
		Закон об информации, защите объектов информатизации - определение информации -> сведения вне зависимости от формы представления.

		\term{Информация} - совокупность данных и соответствующих методов её обработки.

		\term{Информация} - сведения, необходимые для принятия решения (третий вариант определения).

		\term{Сообщение} - совокупность зарегистрированных данных.

		\term{Информация} есть информация, а не материя или энергия.

		\term{Информационная сфера} -  информация, информационная инфраструктура, информационные объекты (носители, потребители), совокупность регламентирующих отношений.
		\imgnh{IMG/1/21},0.8

		Часто информационные объекты разделяют на критически важные, важные и остальные объекты

		Критически важный информационный объект - информационный объект, вывод из строя которого приводит к выходу из строя информационной системы

		Важный информационный объект - информационный объект, вывод из строя которого  приводит к утрате информационной системой некоторых её функций.

		Иной информационный объект - информационный объект, вывод из строя которого не влечёт существенных изменений для информационной системы

		Информационный объект содержит связанные ресурсы, структуры и программы

		\imgnh{IMG/1/20},0.8

		\term{Информационная безопасность} - такое состояние рассматриваемой информационной системы, при котором она, с одной стороны, способна противостоять дестабилизирующему воздействию внешних и внутренних информационных угроз, а с другой стороны, её функционирование не создаёт информационных угроз окружающей среде.

		\subsubsection{Угрозы}

			\term{Угроза} - совокупность факторов и условий, создающих опасность нарушения ИБ организации, вызывающую (или способную вызвать) негативные последствия для организации (ГОСТ 53114-2008)
			\imgnh{IMG/1/01},0.8

			фото - угроза нарушения конфиденциальности

			\term{Канал утечки по времени} (НСД)
			\imgnh{IMG/1/02},0.8
			\imgnh{IMG/1/22},0.8

			\term{Угроза нарушения целостности информации} - незаконные модификация или уничтожение информации.
			\imgnh{IMG/1/03},0.8

		\subsubsection{Классификация угроз по природе происхождения}
			\imgnh{IMG/1/04},0.8

		\subsubsection{Каналы НСД}
			\imgnh{IMG/1/05},0.8
			Классификация взято из учебника (забытого =)

		\subsubsection{Уязвимость ИС}
			\imgnh{IMG/1/06},0.8
			Отличие от угрозы в том, что угроза характеризует внешнюю среду, а уязвимость характерна для конкретной системы.
			\term{Уязвимость} - свойство, обусловливающее возможность реализации угроз безопасности для обрабатываемой информации.

		\subsubsection{Оценка угроз и уязвимостей:} прямая экспертная оценка, статистический анализ, факторный анализ

		\subsubsection{Экспертные оценки}
		Методы поиска решений не поддающихся формализации задач, основанные на суждениях (оценках) экспертов.
		\imgnh{IMG/1/07},0.8

			Способы работы с экспертами:
			\begin{itemize}
				\item Интервьюирование
				\item Анкетирование
			\end{itemize}

			Форма выражения оценки может быть явной  и неявной

			Неявное выражение состоит в том, что эксперт ранжирует оцениваемые элементы (объекты, явления) по степени их важности. Возможен вариант выделения групп с последующим ранжированием объектов внутри группы.

			Явная оценка - эксперт даёт элементам лингвистические или количественные оценки (напр, опасно / безопасно / очень опасно).
			Оценка считается согласованной, если коэффициент конкордации $W$ (коэфф.  согласованности экспертов) больше 0.75

			$$
			W = \frac{12}{d^2(m^3 - m)}S
			$$
			\imgnh{IMG/1/08},0.8

			В случае, когда в ранжировке есть одинаковые значения - применяют другую формулу (некоторые факторы оценены на одинаковый ранг)
			$$
			W = \frac{12 S}{d^2(m^3 - m) - d \sum\limits^d_{s=1}T_s}
			$$
			\imgnh{IMG/1/09},0.8
			\imgnh{IMG/1/10},0.8

			Если с согласованностью проблемы - может применяться метод делфи (aka обратная связь) - эксперты знакомится с анонимными аргументированными обоснованиями суждений коллег.

	\subsection{Оценка привлекательности активов для потенциального злоумышленника}
		\imgnh{IMG/1/11},0.8
		$$
		\gamma = \frac{P^U B_0}{C_0}
		$$
		$C_0$ - стоимость затрат злоумышленника на реализацию угрозы
		$B_0$ - профит злоумышленника при реализации угрозы
		$P^U$ - вероятность реализации угрозы

		\imgnh{IMG/1/12},0.8
		Конечная формула
		\imgnh{IMG/1/13},0.8

	\subsection{Методика оценки уязвимости}
		Малюк, Герасименко

		\imgnh{IMG/1/14},0.8

		Условия НСД -
		\begin{itemize}
			\item Нарушитель должен получить доступ в контролируемую зону
			\item Во время нахождения нарушителя в КЗ должен проявиться соответствующий канал НСД
			\item Нарушитель должен обладать средствами для использования канала НСД
			\item В канале НСД в момент  доступа нарушителя должна быть защищаемая информация.
		\end{itemize}
		Эти события случайны.
		%15
		\imgnh{IMG/1/16},0.8

		ГосТехКомиссия - 4 категории нарушителя - Неподготовленный( пользователей) -> Разработчик (Максимальный доступ и возможности)

		\imgnh{IMG/1/17},0.8
		$L$ - количество зон;
		$P^{\mbox{б}}$ - базовая вероятность реализации угрозы (?)

		Пусть $K^*$ есть интересующее нас подмножество из полного множества потенциально возможных нарушителей. Тогда величина $P_{ij\{K^*\}}$ (третья формула) есть ничто иное, как вероятность нарушения защищённости информации указанным подмножеством нарушителей по $j$-м фактору в $i$-м компоненте системы
		\begin{itemize}
			\item $P_{ij\{K^*\}}$		Наиболее опасный нарушитель
			\item $P_{ik\{J^*\}}$		Наиболее небезопасный канал
			\item $P_{jk\{I^*\}}$		Наиболее уязвимый компонент
		\end{itemize}

		REF: DSEC классификация угроз

		\subsection{Оценка риска ИБ}
		\imgnh{IMG/1/18},0.8

		Риск - вероятность реализации угрозы и нанесение ущерба

	\subsection{Семинар}
		\imgnh{IMG/1/19},0.8

		Лекция - через НЕЧ неделю
		На следующей НЕЧ - вопросы семинара

		Пункт 2 - индивидуальный задания на зачёт.
		1 половина  - ПК
		2 половина - Смартфон
		(по Коновалову)
		Итог  - таким  образом видно, что вероятность ( угроза является более/менее актуальной)

		Время - 15-00


	%270916
	\section{Понятие формальной модели безопасности}
	Основные виды доступа (на чтение и на запись):
	\imgnh{IMG/2/01},0.8
	
	\subsection{Монитор безопасности обращений}
	\imgnh{IMG/2/02},0.8
	Требования к монитору безопасности обращения
	\begin{itemize}
		\item Ни один запрос на доступ не должен проходить в обход МБО
		\item Работа монитора безопасности обращений должна быть легко верифицируема
	\end{itemize}

	\subsection{Модель Харриссона-Рузо-Ульмана (HRU)}
	%Фото 3
	\imgnh{IMG/2/03},0.8

	Множество объектов больше числа субъектов, так как неактивные субъекты попадают в множество объектов.

	\subsubsection{1-я базовая операция - добавление права}
	\begin{verbatim}
	enter r into M[s,o] (S \in S,  )
	\end{verbatim}
	\imgnh{IMG/2/04},0.8
	Изменяется только выбранная ячейка $[S,O]: M'[s,o] = M[s,o]\cup\{r\}$; остальные ячейки остаются неизменными.

	\subsubsection{2-я базовая операция - удаление права}
	\begin{verbatim}
	delete r from M[s,o] (S \in S, o \in O)
	\end{verbatim}
	\imgnh{IMG/2/05},0.8
	Изменяется только выбранная ячейка $[S,O]: M'[s,o] = M[s,o]\backslash \{r\}$; остальные ячейки остаются неизменными.

	\subsubsection{3-я базовая операция - создание субъекта}
	\begin{verbatim}
	Create subject s (S \notin S)
	\end{verbatim}
	\imgnh{IMG/2/06},0.8
	Добавляем по строке/столбцу в множества объектов и субъектов. Все элементы этих строк пусты.


	\subsubsection{4-я базовая операция - удаление субъекта}
	\begin{verbatim}
	destroy subject s (S \in S)
	\end{verbatim}
	\imgnh{IMG/2/07},0.8
	Удаляем по строке/столбцу из множества объектов и субъектов.

	\subsubsection{5-я базовая операция - создание объекта}
	\begin{verbatim}
	Create object o (o \notin O)
	\end{verbatim}
	\imgnh{IMG/2/08},0.8
	Добавляем по элементу в множества объектов. Все элементы столбца пусты.

	\subsubsection{6-я базовая операция - удаление объекта}
	\begin{verbatim}
	destroy object o (o \in O)
	\end{verbatim}
	\imgnh{IMG/2/09},0.8
	Удаляем столбец из множества объектов.

	\subsubsection{Задачи}
	\term{Задача 1.} Создать субъект, передать ему право владения объектом O.
	\begin{lstlisting}
	Program alfn[s,o]
	create subject s1;
	enter own into M[s,o];
	\end{lstlisting}

	\term{Задача 2.} Перевести матрицу из состояния 1 в 2;
	\imgnh{IMG/2/10},0.8
	\begin{lstlisting}
	Program task2[s,o]
	create subject s2;
	// - считаем, что создан вместе с субъектом s2 (create object o3);
	delete W from M[o2,s1];
	enter W into M[s2,o1];
	enter R,W into M[s2,o2];
	enter W into M[s1,o3];
	\end{lstlisting}

	\term{Задание, которое не будет проверяться}: Четырём субъектам присвоено право на владение объектом

Написать программу, которая создаст новый субъект, и добавит право чтения тех объектов, к которым у кого-либо имеется права владения (own)

	\imgnh{IMG/2/11},0.8

	\subsection{Модель take-grant}
	\imgnh{IMG/2/12},0.8
	\imgnh{IMG/2/13},0.8

	Состояния изменяются под воздействиями 4-х видов
	\begin{itemize}
		\item Команда \term{take} - <<Брать>>: take(a,x,y,z); (право a; субъектом x; у субъекта y; по отношению к объекту z)
		      \imgnh{IMG/2/14},0.8
		\item Команда \term{grant} - <<Давать>>: grant(a,x,y,z); (право a; субъектом x; у субъекта y; по отношению к объекту z)
		      \imgnh{IMG/2/15},0.8
		\item Команда \term{create} - <<Создать>>: grant($\beta$,x,y); (x создаёт объект y с правами доступа на него $\beta \subseteq R, y$ - новый объект)
		      \imgnh{IMG/2/16},0.8
		\item Команда \term{remove} - <<удалить>>: remove(a,x,y); (x удаляет права доступа на объект y)
		      \imgnh{IMG/2/17},0.8
		     Если удалить все права, можем считать субъект удалённым.
	\end{itemize}

	Одним из основных вариснтов использования модели является анализ на возможность утечки прав доступа.
	
	\imgnh{IMG/2/18},0.8
	Безопасность системы рассматривается с точки зрения возможности получения субъектом прав доступа к определённому объекту (при этом в начальном состоянии такие права отсутствуют) при определённой кооперации субъектов путём последовательного изменения состояния системы на основе выполнения команд.

	Предметом анализа при этом являются установленные в начальный момент времени отношения между субъектами по получению и передаче прав доступа на объекты системы, и возможные ограничения на дальнейшую кооперацию субъектов в процессе функционирования системы.

	Не получится ли так, что какой-либо субъект не получит права, которые он получить не должен?

	\subsubsection{Команды модели Take-Grant}
	\imgnh{IMG/2/19},0.8

	\term{Санкционированный доступ}
	\imgnh{IMG/2/20},0.8
	Задание на фотке с графом
	\imgnh{IMG/2/21},0.8
	\imgnh{IMG/2/22},0.8
	\imgnh{IMG/2/23},0.8
	\imgnh{IMG/2/24},0.8
	
	\subsubsection{Определение распределённой компьютерной системы}
	\imgnh{IMG/2/25},0.8

	Способы обособления подмножеств субъектов и объектов
	\begin{itemize}
		\item Группирование
		\item Локализация
		\item Присвоение идентификатора
	\end{itemize}

	\subsubsection{Типы распределённых компьютерных систем с точки зрения безопасности}
	\imgnh{IMG/2/26},0.8

	Структура информационного потока в распределённых компьютерных системах
	\imgnh{IMG/2/27},0.8

	\term{Определение 2} Удалённым доступом $p^{out} = \mbox{Stream}(S_m,o_i) \rightarrow O_j$ субъекта $S_m$ пользователя в локальном сегомнте $\Lambda_1$ к объекту $o_j$ в локальном сегменте $\Lambda_2$ называется порождение субъектом $s_m$ через телекоммуникационные субъекты $s^{(1)}_t$ и $s^{(2)}_t$ локальных сегментов $\Lambda_1$ и $\Lambda_2$ потока информации между (далее - см. слайд).

	Мы можем говорить о внутризональном разграничении доступа и межзональных правилах.
	\imgnh{IMG/2/28},0.8

	\term{Определение 3} Зоной в распределённой КС называется совокупность подмножества пользователей, подмножества объектов доступа и подмножества пользователей физических объектов, обособленных в локальный сегмент с отдельной (внутризональной) политикой безопасности.

	\subsection{Зональная модель разграничения доступа}
	\imgnh{IMG/2/29},0.8
	Есть внутризональная политика безопасности и межзональная политика безопасности. Внутризональная политика безопасности обеспечивается соответствующим монитором безопасности.
	
	Внутризональный монитор безопасности (определение)
	\imgnh{IMG/2/30},0.8

	Зональная модель разграничения доступа
	\imgnh{IMG/2/31},0.8

	Для того, чтобы получить доступ к внешнему объекту необходимо выполнить вход в зону через межзональную политику, после чего осуществлять запросы через внутризональную политику зоны, в которой запрашиваем объект.
	\imgnh{IMG/2/32},0.8

	Зональная модель разграничения доступа
	\imgnh{IMG/2/33},0.8

	\imgnh{IMG/2/34},0.8

	10 числа пары нет, селед пара - семинар через НЕЧ

	След лекция - теория распознавания образов - смотреть теорвер.

\end{document}